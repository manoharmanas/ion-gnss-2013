\documentclass[twocolumn,10pt]{article}

%%%%%%%%%%%% PACKAGES %%%%%%%%%%%%%%%%%%%%%%%%%%%%%%%%%%%%%%%%%%%%%%%%%%%%%%%%%%

\usepackage[pdftex]{graphicx}
  \graphicspath{../graphics}
  \DeclareGraphicsExtensions{.pdf,.jpeg,.png}
\usepackage{cite}
\usepackage[cmex10]{amsmath}
  \interdisplaylinepenalty=2500

\usepackage[sf,sl,outermarks]{titlesec}
  \titleformat{\section}
    {\bfseries\MakeUppercase}{\thesection}{1em}{}
  \titlespacing{\section}
    {0pc}{\baselineskip}{0.5ex}

%%%%%%%%%%% STYLE SETTINGS %%%%%%%%%%%%%%%%%%%%%%%%%%%%%%%%%%%%%%%%%%%%%%%%%%%%%

\hyphenation{op-tical net-works semi-conduc-tor}
\bibliographystyle{IEEEtran}

%% font stuff
\usepackage{mathptmx}
\renewcommand{\rmdefault}{ptm}

%% geometry stuff %%
\setlength{\textheight}{8.75in}
\setlength{\columnsep}{2.0pc}
\setlength{\textwidth}{6.8in}
% \setlength{\footheight}{0.0in}
\setlength{\topmargin}{0.25in}
\setlength{\headheight}{0.0in}
\setlength{\headsep}{0.0in}
\setlength{\oddsidemargin}{-.19in}
\setlength{\parindent}{0pc}
\setlength{\parskip}{\baselineskip}

%%%%%%%%% DOCUMENT BEGINNING %%%%%%%%%%%%%%%%%%%%%%%%%%%%%%%%%%%%%%%%%%%%%%%%%%%

\begin{document}

%%%%%%%%%%%%%%%%%
%% Top Matters %%
%%%%%%%%%%%%%%%%%

\title{\textbf{An On-Line Visual Driver Aid for Safe and Precise Convoy Following in Visibility-Impaired Conditions}}
\author{
  Robert Cofield, Scott Martin and David Bevly \\
  \em{GPS \& Vehicle Dynamics Laboratory (GAVLab)} \\
  \em{Auburn University} \\
  % Email: robertgcofield@auburn.edu
}
\date{} %% kill date
\maketitle

%%%%%%%%%%%%%%
%% Abstract %%
%%%%%%%%%%%%%%

\begin{abstract}
  % good intro of what is being presented.
  A driver aid utilizing GNSS positioning is presented which will augment or replace visual cues in scenarios where line-of-sight is unreliable or unavailable, regardless of the ground-based deployment platform.

  % positioning algorithms used
  Dynamic-base Real-Time Kinematic (DRTK) GPS is used to find a relative position from the leader to the follower, while the Time Differenced Carrier Phase (TDCP) measurement is used to compute the change in position of both vehicles.

  % Both of these measurements are recorded over time to yield the path taken by the leader relative to the follower with very high accuracy. The results are refined to preserve only information immediately relevant to the following task and presented on a Graphical User Interface (GUI). Relevant information may comprise distance to the nearest path point as measured along the vehicle's lateral axis (“path deviation”), curvilinear path spacing to adjacent lead vehicle, present velocity, and any associated risk factor. The primary risk factor examined is rear-end collision, which becomes increasingly probable during low-visibility conditions.


  % For a given platform, vehicle dynamic characteristics will greatly influence the desired spacing and ground speed, as well as the ability to minimize lateral deviation from the path of a leader with dissimilar dynamics. Accordingly, a set of driver-input scalar parameters necessary for platform and scenario independence is determined by examining driver performance using the GUI with several 
  
  % From these trials, features to accept corresponding driver inputs are incorporated into the GUI such that modification is possible while driving without presenting a distraction from the driving task. This necessitates an interface view which is clean and devoid of non-essential information. Once this has been accomplished, safety warnings are refined without risk of damaging equipment. Ranges of acceptable braking are input by the driver for various safety hazard levels. Upon crossing one of these predetermined thresholds given instantaneous velocity or curvilinear following distance, a negative longitudinal acceleration violating a corresponding limit would be required to avoid collision with the leader. The appropriate warning is relayed to the driver warning of the risk.

  % Testing is then performed in multiple platforms of varied dynamic characteristics to evaluate fidelity gains with respect to two variables: direct line-of-sight and visible cues. A performance improvement test is first conducted in environments benign to the following task, in which drivers are asked to perform a similar task both with and without the aid of the GUI presented. In this manner, fidelity gains are evaluated when the presence of lane markings and other visually distinct features make path replication much simpler. Another set of tests is then presented with terrain cues removed, similar to a desert driving scene without any objects or markings useful for relative localization. Line-of-sight is maintained throughout this test as well. Scenarios are then constructed which make high-fidelity path replication extremely difficult to impossible without aid; drivers are asked to replicate an intricate path without any line-of-sight or visual markers, using the GUI for navigation. In all test cases, simulated and actual, the metric by which driver performance is scored is based upon drivers' minimization of lateral deviation and ability to keep constant following distance goals. In instances where erratic paths result in rapidly changing ground speeds, a constant spacing may not be practical. In these instances, it is expected that a following driver should maintain a velocity equal to the leader's velocity at the same position. All of these qualifiers are analyzed in detail during post-processing. In this way the GUI will be shown to be significantly beneficial to aiding convoy drivers in high precision following.

\end{abstract}

%%%%%%%%%%%%%%%
%% Biography %%
%%%%%%%%%%%%%%%

\section*{Biography}

  Robert Cofield is investigating navigation and positioning of ground and other vehicles using GNSS and fused sensor solutions. He works in the Auburn University GPS \& Vehicle Dynamics Laboratory.

%%%%%%%%%%%%%%%%%%
%% Introduction %%
%%%%%%%%%%%%%%%%%%

\section*{Introduction}

  The use of vehicle convoys is common in transportation, where safety and path precision are frequently desired.  Extremely close spacing can be used to reduce fuel costs, but introduces a collision risk.  In other cases, when traveling over a path that has unstable or dangerous areas nearby, a following vehicle attempts to stay within the track of a vehicle ahead. When visibility is impaired, the ability of any convoy driver to adhere to a path of known safety, and the ability to avoid colliding with the leader is significantly reduced. Real-time path fidelity information is sought which will enable the following driver to intuitively and safely maintain a high-fidelity execution of the desired path during situations in which the leader is either not directly visible or close enough that the risk of collision is imminent.


%%%%%%%%%%%%%%%%%%%%%%%%%%%%%%%
%% Conclusions / Future Work %%
%%%%%%%%%%%%%%%%%%%%%%%%%%%%%%%

\section*{Conclusions \& Future Work}


%%%%%%%%%%%%%%%%%%%%%%%%
%%%%% Bibliography %%%%%
%%%%%%%%%%%%%%%%%%%%%%%%

\nocite{ScottThesis}
\bibliography{../bib/master}


\end{document}