\documentclass[twocolumn,10pt]{article}

%%%%%%%%%%%% PACKAGES %%%%%%%%%%%%%%%%%%%%%%%%%%%%%%%%%%%%%%%%%%%%%%%%%%%%%%%%%%

\usepackage[pdftex]{graphicx}
  % \graphicspath{../graphics}
  % \DeclareGraphicsExtensions{.pdf,.jpeg,.png}
\usepackage{cite}
\usepackage[cmex10]{amsmath}
\usepackage[sf,sl,outermarks]{titlesec}
\usepackage{caption}
\usepackage{mathptmx}
\usepackage{titling}

%%%%%%%%%%% STYLE SETTINGS %%%%%%%%%%%%%%%%%%%%%%%%%%%%%%%%%%%%%%%%%%%%%%%%%%%%%

\hyphenation{op-tical net-works semi-conduc-tor}
\bibliographystyle{IEEEtran}

%% section title formatting
\titleformat{\section}  {\bfseries\MakeUppercase}{\thesection}{1em}{}
\titlespacing{\section}  {0pt}{\baselineskip}{0pt}
%% subsection title formatting
\titleformat{\subsection}  {\bfseries\normalsize}{\thesubsection}{1em}{}
\titlespacing{\subsection}  {0pt}{0pt}{-5pt}

\pagenumbering{gobble} % turn off page numbering

%% font stuff
\renewcommand{\rmdefault}{ptm}
\captionsetup[figure]{labelfont={small},textfont={it,footnotesize}}
\captionsetup[table]{labelfont={small},textfont={it,footnotesize}}

\interdisplaylinepenalty=2500

%% geometry stuff %%
\setlength{\droptitle}{-.6in} % title position
\setlength{\textheight}{9.25in}
\setlength{\columnsep}{2.0pc}
\setlength{\textwidth}{6.8in}
\setlength{\topmargin}{0.25in}
\setlength{\headheight}{0.0in}
\setlength{\headsep}{-0.4in}
\setlength{\oddsidemargin}{-.19in}
\setlength{\parindent}{0pc}
\setlength{\parskip}{\baselineskip}

%%%%%%%%% DOCUMENT BEGINNING %%%%%%%%%%%%%%%%%%%%%%%%%%%%%%%%%%%%%%%%%%%%%%%%%%%

\begin{document}

%%%%%%%%%%%%%%%%%
%% Top Matters %%
%%%%%%%%%%%%%%%%%

\title{\textbf{An On-Line Visual Driver Aid for Safe and Precise Convoy Following in Visibility-Impaired Conditions}}
\author{
  Robert Cofield, Scott Martin and David Bevly \\
  \em{GPS \& Vehicle Dynamics Laboratory (GAVLab)} \\
  \em{Auburn University} \\
  % Email: robertgcofield@auburn.edu
}
\date{} %% kill date
\maketitle

%%%%%%%%%%%%%%
%% Abstract %%
%%%%%%%%%%%%%%

\begin{abstract}

  % good intro of what is being presented.
  A driver aid utilizing GNSS positioning is presented which will augment or replace visual cues in scenarios where line-of-sight is unreliable or unavailable, regardless of the ground-based deployment platform.
  % positioning algorithms used
  Dynamic-base Real-Time Kinematic (DRTK) GPS is used to find a relative position from the leader to the follower, while the Time Differenced Carrier Phase (TDCP) measurement is used to compute the change in position of both vehicles over time.

  % Both of these measurements are recorded over time to yield the path taken by the leader relative to the follower with very high accuracy. The results are refined to preserve only information immediately relevant to the following task and presented on a Graphical User Interface (GUI). Relevant information may comprise distance to the nearest path point as measured along the vehicle's lateral axis (“path deviation”), curvilinear path spacing to adjacent lead vehicle, present velocity, and any associated risk factor. The primary risk factor examined is rear-end collision, which becomes increasingly probable during low-visibility conditions.


  % For a given platform, vehicle dynamic characteristics will greatly influence the desired spacing and ground speed, as well as the ability to minimize lateral deviation from the path of a leader with dissimilar dynamics. Accordingly, a set of driver-input scalar parameters necessary for platform and scenario independence is determined by examining driver performance using the GUI with several 
  
  % From these trials, features to accept corresponding driver inputs are incorporated into the GUI such that modification is possible while driving without presenting a distraction from the driving task. This necessitates an interface view which is clean and devoid of non-essential information. Once this has been accomplished, safety warnings are refined without risk of damaging equipment. Ranges of acceptable braking are input by the driver for various safety hazard levels. Upon crossing one of these predetermined thresholds given instantaneous velocity or curvilinear following distance, a negative longitudinal acceleration violating a corresponding limit would be required to avoid collision with the leader. The appropriate warning is relayed to the driver warning of the risk.

  % Testing is then performed in multiple platforms of varied dynamic characteristics to evaluate fidelity gains with respect to two variables: direct line-of-sight and visible cues. A performance improvement test is first conducted in environments benign to the following task, in which drivers are asked to perform a similar task both with and without the aid of the GUI presented. In this manner, fidelity gains are evaluated when the presence of lane markings and other visually distinct features make path replication much simpler. Another set of tests is then presented with terrain cues removed, similar to a desert driving scene without any objects or markings useful for relative localization. Line-of-sight is maintained throughout this test as well. Scenarios are then constructed which make high-fidelity path replication extremely difficult to impossible without aid; drivers are asked to replicate an intricate path without any line-of-sight or visual markers, using the GUI for navigation. In all test cases, simulated and actual, the metric by which driver performance is scored is based upon drivers' minimization of lateral deviation and ability to keep constant following distance goals. In instances where erratic paths result in rapidly changing ground speeds, a constant spacing may not be practical. In these instances, it is expected that a following driver should maintain a velocity equal to the leader's velocity at the same position. All of these qualifiers are analyzed in detail during post-processing. In this way the GUI will be shown to be significantly beneficial to aiding convoy drivers in high precision following.

\end{abstract}

%%%%%%%%%%%%%%%
%% Biography %%
%%%%%%%%%%%%%%%

\section*{Biography}

  Robert Cofield is investigating navigation of ground and other vehicles using GNSS and fused sensor solutions.  He works in the Auburn University GPS \& Vehicle Dynamics Laboratory.

%%%%%%%%%%%%%%%%%%
%% Introduction %%
%%%%%%%%%%%%%%%%%%

\section*{Introduction} \label{sec:intro}

  The use of vehicle convoys is common in transportation, where safety and path precision are frequently desired.  Extremely close spacing can be used to reduce fuel costs, but introduces a collision risk.  In other cases, when traveling over a path that has unstable or dangerous areas nearby, a following vehicle attempts to stay within the track of a vehicle ahead.  When visibility is impaired, the ability of any convoy driver to adhere to a path of known safety, and the ability to avoid colliding with the leader is significantly reduced.  Real-time path fidelity information is sought which will enable the following driver to intuitively and safely maintain a high-fidelity execution of the desired path during situations in which the leader is either not directly visible or close enough that the risk of collision is imminent.

%%%%%%%%%%%%%%%%%%%%%%%%%%%%%%%%
%% Previous Work & Literature %%
%%%%%%%%%%%%%%%%%%%%%%%%%%%%%%%%

\section*{Literature} \label{sec:lit}

  This work follows and builds upon that done by Martin in \cite{ScottThesis} and Travis in \cite{travisdiss}. 

%%%%%%%%%%%%%%%%%
%% GUI Display %%
%%%%%%%%%%%%%%%%%

\section*{Graphical User Interface} \label{sec:gui}

  \begin{figure}[ht] \centering
    \includegraphics[width=\columnwidth] {../graphics/final_design_data.png}
    \caption{Qt GUI in normal operation} \label{fig:qt_normal}
  \end{figure}

  \begin{figure}[ht] \centering
    \includegraphics[width=\columnwidth] {../graphics/earth_slow.png}
    \caption{Google Earth GUI alerting user to slow and correct leftward} \label{fig:earth_alerts}
  \end{figure}

%%%%%%%%%%%%%%%%%%%%%%%%%%%%%%%%%%%%%%%%%%
%% Experimentation Procedures & Results %%
%%%%%%%%%%%%%%%%%%%%%%%%%%%%%%%%%%%%%%%%%%

\section*{Experimentation} \label{sec:exper}

  Experimental trials were conducted using three different testing scenarios: a lane-change test, a precision following test, and a zero landmark test.  

  %% Lane-Change Test
  \subsection*{Lane Change Test}

    To produce binary pass-fail results, a maneuver common in typical roadway driving, the lane change, was used in a situation in which the maneuver could be replicated properly or improperly to produce a binary result.  This test takes place on the National Center for Asphalt Technology (NCAT) test track in Opelika, AL, which is a two lane, 1.7 mile oval with turns comparable to those found on an interstate highway.  The leader drives through a $180^{\circ}$ turn and upon exiting makes a lane change between any of six cones spaced $10~m$ apart along the center stripe at the start of the straightaway.  This event is visually obscured from the following driver, who has no foreknowledge of which cone pair will be chosen, so there is a 20\% chance of choosing the correct pair simply by guessing.

  %% Precision Following Test %%
  \subsection*{Precision Following Test}

    % motivation
    The lane change replication test is one example of implementing the path duplication tool, but does not produce the detailed results necessary for a formal conclusion favoring the usefulness of one GUI over the other.  Centimeter-level measurements are available, so it is of great interest to determine whether either tool enables a convoy driver to carry out the following task with this improved level of precision.
    
    % description
    The precision following test begins and ends with both vehicles parked atop the center stripe of the NCAT test track.  The lead vehicle accelerates to approximately 45 mph then begins a sinusoidal path with a mean about the center stripe, a period of approximately $10~s$, and an amplitude that puts the wheels of the lead vehicle upon the outermost lane marking at the peaks.  Once reaching 45 mph, the magnitude of the leader’s ground plane velocity vector will vary according to position on the track.  Along the two $180^{\circ}$ turns it will be approximately 45 mph, and along the straightaways it will be approximately 65 mph.  Throughout the test, the following driver is attempting to maintain a inter-vehicular spacing as low as possible without incurring any distance alerts, and accumulate as little deviation over time as possible.
    
    \begin{figure}[ht] \centering
      \includegraphics[width=\columnwidth]{../graphics/precision_following_alert_percents.png}
      \caption{Percentages of best runs from the precision following test in which alerts were incurred}
      \label{fig:precision_following_alert_percents}
    \end{figure}

    % discussion
    This test primarily focused on distinguishing which GUI best provided aid in path duplication; for a comparative analysis, the test was conducted with the aid of each GUI individually, then without any assistance information at all.  As this test was conducted in an environment equivalent to most American two-lane roads and the leader does not stray from the two outer lane markings, it was anticipated that the lateral deviation should be bounded by approximately the width of the road minus the track width of the widest vehicle, or $5.8~m$.  This allows for scrutiny of the deviation in greater detail than when the lateral deviation is not bounded by design.

    \begin{table}[htbp] \centering \caption{Mean absolute deviation for the precision following test}
    \begin{tabular}{r|c} 
        GUI&    Mean abs. dev. \\ \hline\hline
        Earth&      0.9677 m \\
        Qt&   0.4719 m \\
        Control&    0.2069 m \\ \hline   
    \end{tabular} \label{tab:precision_dev_mean} \end{table}

  %% Zero-Landmark Test %%
  \subsection*{Zero-Landmark Test}

    % motivation
    The precision following test took place in an environment abundant in visual landmarks by which to localize, such as lane markings and road signs.  Given the motivation for these tools, it is necessary to examine the performance of both GUIs in a situation where this type of assistance is denied.  The zero landmark test was therefore constructed to determine the impact of removing visual awareness of surrounds as well as waypoints along the leaders path to assist in replicating it.
    
    \begin{figure}[ht] \centering
      \includegraphics[width=\columnwidth]{../graphics/zero_landmark_alert_percents.png}
      \caption{Percentages of best runs from the zero landmark test in which alerts were incurred}
      \label{fig:zero_landmark_alert_percents}
    \end{figure}

    % test description
    Both vehicles began parked in a large, open expanse of flat asphalt (`skidpad') with excellent satellite visibility.  There are no clearly visible artifacts upon the ground, though some objects are visible at the outer edges of the skidpad.  To further obscure visual following, the test was conducted at night.  The lead vehicle drove in chaotic patterns intended present a path with is difficult to follow while the following vehicle attempted to adhere to the path as well as possible and maintain the closest possible curvilinear separation distance without triggering any alert messages.

    \begin{table}[htbp] \centering \caption{Mean absolute deviation for the zero landmark test}
    \begin{tabular}{r|c} 
        GUI&    Mean abs. dev. \\ \hline\hline
        Earth&      2.7404 m \\
        Qt&   3.6830 m \\
        Control&    0.7267 m \\ \hline   
    \end{tabular} \label{tab:zero_dev_mean} \end{table}


%%%%%%%%%%%%%%%%%%%%%%%%%%%%%%%
%% Conclusions / Future Work %%
%%%%%%%%%%%%%%%%%%%%%%%%%%%%%%%

\section*{Conclusions \& Future Work}

  %% summary
  Two graphical tools were developed to assist drivers in convoys with high fidelity to the lead vehicle’s path while maintaining a safe spacing between them.  Experimentation reveals that they were quite effective in helping to enforce safe curvilinear following distances, while more development is needed to optimize results for lateral path deviation.  

  %% improvements - overlay prediction
  The zero landmark test was by far the most successful in garnering qualitative feedback.  Drivers suggested a projected path overlay which gave a prediction of the follower’s path in realtime given course and yaw rate information.  This would counter the effects of being unable to tell how present actions would effect deviation in the approximately 1 s required for steering input to be reflected in either GUI.  Furthermore, it was suggested, a model prediction scheme as in \cite{williamthesis} could be employed to show not only predicted path over some window, but estimate the current position and overlay it.  Future work will pursue adding these improvements to the Qt GUI and continuing to refine that tool, as it has been shown more successful than the Earth GUI at accomplished the goals outlined herein.


%%%%%%%%%%%%%%%%%%%%%%%%
%%%%% Bibliography %%%%%
%%%%%%%%%%%%%%%%%%%%%%%%

\nocite{CofieldUGThesis}
\bibliography{../bib/master}


\end{document}